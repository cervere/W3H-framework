\documentclass{article}
\usepackage{standalone}
\usepackage{blindtext}
\usepackage{titlesec}
\usepackage[english]{babel}
\usepackage[utf8]{inputenc}
\usepackage{pdfpages}
\usepackage{hyperref}
%\usepackage{fancyhdr}

\title{Survival}
\author{and what it takes...}
\date{ }

\addto\captionsenglish{
    \renewcommand*\contentsname{Contents}
}

\newcommand\invisiblesection[1]{%
  \refstepcounter{section}%
  \addcontentsline{toc}{section}{\protect\numberline{\thesection}#1}%
  \sectionmark{#1}}
  
\begin{document}
\maketitle
\tableofcontents
\clearpage

\section{Outline}
Computational neuronal modeling of the synergy between different mem-
ory systems in animal brain, responsible for Autonomous Learning to
survive in the unknown external world.  Exploring the possibilities of
applying the key underlying principles in modern day Machine Learning
algorithms. The plan is to use Microsoft Malmo, an AI experimentation
platform, as a virtual environment with an agent motivated to survive in
it. This platform allows to construct the external world and to be able
to train and test the neural models on the agent.

\section{Hypothesis}
%\includepdf[pages=1-]{coverletter.pdf}
The preliminary idea about the neural pathways involved and the role they play in helping the animal to survive has been drafted by Frederic Alexandre in \href{https://hal.inria.fr/hal-01246653/}{A behavioral framework for a systemic view of brain modeling}.

We try to build a framework that would integrate different memory systems to coordinate together to help the animal survive. Various principles govern the way the framework works. Few of them are:
\begin{itemize}
\item Embodiment
\item Value
\begin{itemize}
\item Emotional
\item Motivational
\end{itemize} 
\item Goals
\item Uncertainty
\item Sensori-motor loops
\item Rules
\item Error prediction (Learning)
\end{itemize} 

\clearpage
\section{Quick notes from Reddish}
\par Interesting - value, not just independent representation, but 
value of taking an action given that the agent
is in a certain situation (or state).
\\
\subsection{State}
\begin{itemize}
\item Categorization of the agent’s current situation as a member of a class of similar situations. 
\item Representative collection of salient observations that might include notable events, environmental configurations, actions, et cetera. 
\item States can include both spatial and temporal extents. 
\item Each unique state is associated with a value representing the time-discounted future reward that a behaving animal would expect when starting from that state.
\end{itemize}

\subsubsection{Extinction as different state space}
"We propose that the extinction phase entails the development of a new (parallel) state space that can then contain a different value estimate."\\
\textbf{Q} : If extinction was earlier suggested to be an inhibition of the existing association (Maxime's paper) rather than a separate association itself, doesn't this contradict?

\subsubsection{Way to demonstrate splitting of state space}
Two experiments in which an animal is faced with
two contexts, one in which reward is provided in both contexts and
another in which reward is provided in only the first context, the
model splits only the state in the situation in which reward is not
provided in the second context

\subsubsection{Example Model}
\begin{itemize}
\item Context 
\item State identification (binary)
\item Reward magnitude
\item time-since-last-reward (for each environment / condition (or situation?) )
\end{itemize}

\section{COSYNE 2018}

\section{Abstract}

Minecraft, a game that is very close to the real world in many ways, offers endless possibilities for exploration with its rich, immersive world and thus aptly suitable for research in general intelligence. We use Malmo, a platform built on the top of Minecraft, to create a simple scenario that mimics Survival in an unknown external world. In this work, we demonstrate how the agent is able to interact with the environment and act on it, driven by an integrated framework of cortico-cortical activity driven sensori-motor loops. Much evidence has been demonstrated about different sensori-motor loops in \textbf{an animal's brain}, mostly from the sensory cortex (acquiring stimulus information) through respective striatal sub-structures  and finally specific frontal cotical areas. Taking this generic nature of loops into account, we implement \textit{one}: a preliminary motor loop through the the supplementary motor areas for evoking an action, \textit{two}: a preferential loop through the Orbito Frontal Cortex (OFC, widely understood to represent partially observable states of the environment) and \textit{three}: a motivation-guided loop through the Anterior Cingulate Cortex (ACC), influencing the other two loops with the animal's internal needs and motivation. The key feature of the framework is that at any time the action is a result of competition between stimulus driven and goal driven aspects of the above mentioned distributed, generic loops. Thus we propose a comprehensive framework, formalizing the existing biological understanding of the sensori-motor loops. At the implementation level, we consider a more general dynamical system, compatible with biological constraints as opposed to the usual, fairly constrained neuronal computation framework. However, the framework design is compatible with the integration of existing detailed neuronal implementations of the brain structures and pathways involved, thereby providing a means to validate them in a systemic context. The very nature of simplication in the framework raises several very important questions related to different kinds valuation of the stimuli (desired, anticipated and actual) and their resolution, the role of striatum in such computations and the tradeoff between the cost of an action and the expected reward.
\end{document}
